\printbibliography[heading=bibintoc]

\begin{comment}
    % SE BIBLIOGRAFIA MANUALE
    % Per mandare nell’indice generale il titolo della bibliografia del documento per la classe book e report
    %\cleardoublepage
    % fa cominciare la bibliografia in una pagina nuova dispari e assegna alla corrispondente voce nell’indice il numero di pagina corretto
    \phantomsection
    % va dato solo se è caricato anche hyperref, serve affiché la voce Bibliografia nell'indice porti alla pagina della Bibliografia
    \addcontentsline{toc}{chapter}{\bibname}
    
    \printbibliography



    BIBLIOGRAFIA
    Elenco di opere scritte o di altro tipo che di solito occupa una sezione autonoma del documento con un titolo (in genere) omonimo.
    ------------------------------------------------
    RIERIMENTO BIBLIOGRAFICO
    Serie di dati che permette di identificare ed eventualmente reperire un’opera. L’insieme dei riferimenti bibliografici costituisce la bibliografia di un documento.
    
    STILE BIBLIOGRAFICO
    Modalità generale adottata per presentare al lettore i riferimenti bibliografici.
    ------------------------------------------------
    CITAZIONE BIBLIOGRAFICA
    Indicazione sintetica, data nel corpo del documento, che rinvia il lettore a un riferimento bibliografico.
    
    SCHEMA DI CITAZIONE
    Modalità generale adottata per presentare al lettore una citazione bibliografica.
\end{comment}