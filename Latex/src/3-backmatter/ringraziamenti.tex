%---------------------------------RINGRAZIAMENTI
% PRIMA DEL CLU
Alla fine del percorso di questi 3 anni e di un nuovo percorso della magistrale che già è iniziato (e per il quale sono già indietro) desidero ringraziare le persone che mi hanno accompagnato e mi sono state donate in questo cammino e per la grandezza di quanto è accaduto nella mia vita. Perdonatemi se dimenticherò qualcuno!

Vorrei innanzitutto ringraziare la mia \textit{famiglia} che mi è sempre stata vicino, supportato, sostenuto e creduto in me. Grazie \textbf{Mamma}, \textbf{Papà}, \textbf{Lorenzo} e \textbf{Nonna Giovanna} e \textbf{Nonno Gianni} che mi guardi da lassù e grazie anche ai nonni di Bologna \textbf{Ettore} e \textbf{Valeria} e anche agli \textbf{zii}.

Grazie ai miei prof del \textit{liceo} che mi hanno \textit{armato} a sufficienza per affrontare le sfide del mondo e mostrato quanto grande può essere la passione per l'educazione (che è passione per l'uomo!): \textbf{Mara Ferroni}, \textbf{Paolo Giglioli}, \textbf{Enrica Spadanuda}, \textbf{Giovanni Battista Nicotra}, \textbf{Roberto Mastri}, \textbf{Lorenzo Raggi}, \textbf{Francesco Severi}, \textbf{Don Marco Ruffini} e tutti gli altri che ho incontrato al Liceo Malpighi di Bologna.

Grazie anche ai pochi (ma buoni) amici di \textit{GS} Bologna che mi hanno fatto incontrare e conoscere il Movimento prima a scuola poi a GS perché attraverso la loro proposta è passato Qualcosa di molto più grande, loro sono stati strumento e servi: \textbf{Batti}, \textbf{Mike}, \textbf{Alys}, \textbf{Ross}, \textbf{MK}, \textbf{Lea}, \textbf{Lodo}, \textbf{Giudi}, \textbf{Giuse}, \textbf{Fede}, \textbf{Bara}, \textbf{Mens}, \textbf{Sal} e tutti gli altri.

Ringrazio il \textbf{prof Alici} per la grande professionalità e serietà che ha messo nell'assistermi in questo lavoro anche nei momenti più critici e per la grande disponibilità umana che ha nei confronti degli studenti.

\begin{verse}
    \textit{Perché i miei pensieri non sono i vostri pensieri, \\
    le vostre vie non sono le mie vie - oracolo del Signore. \\
    Quanto il cielo sovrasta la terra, \\
    tanto le mie vie sovrastano le vostre vie, \\
    i miei pensieri sovrastano i vostri pensieri. \\
    Come infatti la pioggia e la neve \\
    scendono dal cielo e non vi ritornano \\
    senza avere irrigato la terra, \\
    senza averla fecondata e fatta germogliare, \\
    perché dia il seme al seminatore \\
    e pane da mangiare, \\
    così sarà della parola \\
    uscita dalla mia bocca: \\
    non ritornerà a me senza effetto, \\
    senza aver operato ciò che desidero \\
    e senza aver compiuto ciò per cui l'ho mandata.} \\
    Isaia 55, 8-11
\end{verse}

% ARRIVO AL CLU E VECCHI
Il mio arrivo al \textit{CLU} di Bologna non è stato facile, è stato un incontro-scontro, sicuramente non era come me lo sarei aspettato: era pieno di persone che mi volevano bene, ma in un modo da me inaspettato e grandissimo e sicuramente non secondo la mia misura per cui è stato molto difficile \textit{accogliere} questa modalità nuova.

Grazie \textbf{Don Marco} per la serietà con cui mi hai sempre guardato e per avermi invitato fin da subito a conoscere i più grandi della mia facoltà e a buttarmi e implicarmi in un'esperienza che ha solo fatto più grande la mia vita.

Grazie agli amici più \textit{grandi} che ho conosciuto e stimato fin dal primo istante.

Grazie \textbf{Gigi} per l'unità che vivi e testimoni nella tua vita, sei come un padre per me.

Grazie \textbf{Gae} per avermi accolto fin dall'inizio e avermi buttato dentro (quasi troppo) alla vita di Scienze.

Grazie \textbf{Flex} perché il nostro rapporto, nato nell'ultimo anno, è stato di grande aiuto nelle scelte, nei giudizi e nella quotidianità.

Grazie \textbf{Trava} per testimoniare che una vita dedicata allo studio può non essere brutta (attento a non convertirti alla religione del cibo!).

Grazie \textbf{Mary Narciso} per l'affezione che hai verso le persone a cui vuoi bene e per i nostri dialoghi.

Grazie \textbf{Alby} per esistere, farci ridere e perché ti sei affezionato a me.

Grazie \textbf{Giovo} perché anche se ci siamo visti poco ti sei interessato a me dal primo giorno.

Grazie \textbf{Teresina} per come ti sei presa cura della segreteria e della Comunità di Scienze.

Grazie \textbf{Scaio} per la serietà, semplicità e ironia con cui vivi e mi hai accolto.

Grazie \textbf{Luce} per prendermi in giro, condividere la passione per la politica ed essere serio con me in quello che vivo.

Auguri \textbf{Samma}!

Grazie \textbf{Ciamma} per la compagnia nella vita del Movimento di questi anni e per la fedeltà e passione che ci hai sempre messo senza avere nessun ``ruolo'' o responsabilità.

Grazie \textbf{Frappa} per avere dimostrato che nella legge dei 3 fisici anche l'ansiato ce la può fare. Scherzo, grazie per come mi hai preso a cuore negli ultimi mesi.

Grazie \textbf{IlaF} perché non ci crederai, ma mi manchi anche tu con la tua spensieratezza, carica e semplicità.

Grazie \textbf{Flipper} perché ogni tradizione a Fisica è legge!

Grazie anche agli agrari di ieri: \textbf{Vanno} e le tue paccone, \textbf{Maddi} per le cucine delle convi, \textbf{Pallina} per essere un ghei, Don \textbf{Luca Zappi} eterno capo di agraria, \textbf{Gol} per la schiettezza nel dire le cose e \textbf{Stiv} con cui è nata un'amicizia di una serietà inaspettata.

\begin{quote}
    \emph{Allora, scusatemi, concludo. Due cose nella mia vita sono importanti. La prima è questa: che proprio per quel che vi ho detto, il gusto della vita non è negato a chi sbaglia, ma a chi non ha un senso dell’infinito, del destino, dell’ideale, del Mistero presente, perché allora il problema non è sbagliare o non sbagliare. Il gusto della vita non è negato a chi sbaglia: è negato a chi non ha un nesso con il Destino che fa le cose, con il Mistero presente. \emph{Per cui tutto è un’ipotesi positiva, il tempo che per tutti è sinonimo di decadenza, lavora in positivo.} Se guardo la mia vita, che razza di roba è successa! Dico sempre: se è successo così fino adesso, immaginiamoci cosa succederà nel futuro! Ne vedremo delle belle. È interessante, no? È un’avventura. Ed è esattamente qui il problema, perché la seconda cosa è che se dovessi paragonare la mia vita, come si è svolta (c’è una legge fisica che dice che l’orizzonte si muta mutando il punto di osservazione), userei questa metafora: \emph{la mia vita è come una mongolfiera, più vado, più m’innalzo, più mi impegno, più sono dentro a questa vita, più scopro degli aspetti dell’umano che erano impossibili prima: la capacità di fedeltà, di amicizia, di lealtà, di ripresa, di indomabilità, che non avevo mai pensato prima. Perciò, da ultimo, è una gratitudine. Come ho iniziato, così voglio finire: è una gratitudine che caratterizza la mia vita, perciò non ho paura di darla tutta.}} \\
    Enzo Piccinini, \textit{``Esercizi CLU Rimini''}, 12 dicembre 1998
\end{quote}

% QUESTI ANNI: PERSONE E RINGRAZIAMENTI
Le parole di Enzo descrivono benissimo quello che ho visto accadere alla mia vita in questi anni: una \textit{scoperta} di tutto, di me stesso, degli altri e della realtà tutta. \\


Grazie a tutta la comunità di \textit{Scienze}, a tutti quelli che ho incontrato qui, in questi anni mi sono sentito a casa e non sono stato solo nel cammino. \\


Grazie ai \textit{Grafici} delle Elezioni 2022. È stato il primo servizio grande nei confronti di questa compagnia e ha veramente cambiato molto di me in fatto di disponibilità e sguardo verso il mio tempo.

Grazie \textbf{Justin} per la tua stima, schiettezza e semplicità nei miei confronti e dedizione totale verso il Movimento.

Grazie \textbf{Marta} e \textbf{Linda} perché mi avete accolto e aiutato nel lavoro.

Grazie \textbf{Clara}, \textbf{Ste}, \textbf{Sara}, \textbf{AnnaCia}, \textbf{VeryPery} e a tutti gli altri con cui ho lavorato. \\


Grazie ai \textit{Medici} che quella estate e l'anno dopo mi hanno accolto per i precorsi: \textbf{Nico Grut}, \textbf{Carlos}, \textbf{Paolo Scurti}, \textbf{Capitano}, \textbf{Dame}, \textbf{Scanta}, \textbf{Richi Zandri}, \textbf{Vero Comelli}, \textbf{Eugenio}.

E agli altri amici di medicina.

Grazie \textbf{Leti Gennari} la tua amicizia è stata una scoperta preziosa il secondo anno.

Grazie \textbf{Rachi Villa} la tua cura verso le cose e le persone è sempre stimabile. \\

Grazie \textbf{Mele} perché il tuo sguardo su di me è sempre stato come quello di un padre.

Grazie ai miei compagni di corso \textbf{Alice}, \textbf{Nicola}, \textbf{Elia} e \textbf{Ale Dale} e tanti altri per la vostra disponibilità e aiuto nello studio e amicizia.

Grazie agli agrari di oggi: \textbf{Mace} silenzioso costruttore del Movimento, \textbf{Fillo} dal cuore immenso, \textbf{Frusj} un guerriero della verità dal cuore d'oro, buttati nella vita senza paura, \textbf{Paolo} per la tua schiettezza, \textbf{Ferro} schizzato e con una grande attenzione verso gli altri, \textbf{Calo} e \textbf{Noa}.

Grazie \textbf{Press}, \textbf{Tazio}, \textbf{Jucy}, \textbf{Manu} per essere degli scoppiati. Grazie \textbf{Sir}. Grazie \textbf{Ale Cale} per la tua cucina. Grazie \textbf{Sium} per la serietà con cui hai iniziato a guardarti e per avermelo raccontato. Grazie \textbf{Sugo} per la tua intelligenza e ironia.

Grazie \textbf{Francy Mina} perché è nata un'amicizia di una serietà che non credevo possibile, alla fine le questioni sono le stesse per tutti.

Grazie \textbf{M} per la serietà e la stima che è nata grazie a un solo giorno in tre anni (la laurea di Stiv). È la testimonianza che quando uno è serio con la realtà poi le cose accadono. \\


Grazie all'\textit{appa} \textbf{Barozzi 2023/2024} (Sangio, James, Flipper, Salva, Zizza e Sbrembo e Leti Black) per la disponibilità non scontata con cui mi avete accolto.

Grazie \textbf{Maniz} per essere stato una gran matricola. Non aver paura a chiedere e non restare solo.

Grazie \textbf{Salva} per essere sempre stato te stesso nelle cose che ti corrispondono o no.

Grazie \textbf{James} per come mi hai preso a cuore quest'anno in maniera per me totalmente inaspettata e per la fedeltà nell'amicizia e serietà che hai col Movimento.

Grazie \textbf{Sangio} per la nostra amicizia e la semplicità con cui possiamo parlare di tutto. Non aver paura a dare tutto per questa amicizia. \\


Grazie a \textbf{Ciuco}: per quel poco che sono venuto a coro e i dialoghi che abbiamo avuto hai sempre avuto per me una grandissima serietà e cura soprattutto rispetto a quello che vivo.

Grazie \textbf{Pass Leto} per la cura che hai verso i chierichetti e la segreteria. La tua conversione è stato un miracolo che nessuno aveva previsto.

Grazie \textbf{Chiara Z}, \textbf{Ale Giorgini}, \textbf{Mary Santini}, \textbf{ChiaraP}, \textbf{Giudi}, \textbf{Vero}, \textbf{Ayesha} per tutte le domande che fai, \textbf{Benny Cesa} e \textbf{IreBazz}.

GRANDE \textbf{Panna}! Grazie per la compagnia di queste settimane e di questi 3 anni.

Grazie \textbf{Bea}, \textbf{ChiaraL}, \textbf{Marty} e \textbf{White Leti} siete state delle ottime vicine di un esterno e non avete fatto deprimere i maschietti.

Grazie \textbf{Emma} per esserti fidata di me quando ti ho fatto conoscere Scienze, non smettere di scommettere sugli amici che hai incontrato se ci stai bene.

Grazie \textbf{Anna Argelli} per la freschezza e amicizia che hai portato tra i musicisti e le tue amiche. Non avere paura di far vedere chi sei veramente.

Grazie \textbf{Marta} per la nostra amicizia rinata. Non abbatterti e fidati dei tuoi amici.

Grazie \textbf{Saad} per i nostri dialoghi in treno. Sono contento che almeno ad una facoltà ti sei affezionato (Geco).

Grazie \textbf{JLo} per la silenziosa amicizia e affezione di questi anni.

Grazie \textbf{Base} e \textbf{Lelly} per essere anche voi degli scoppiati, \textbf{Benny Peroz} per la cura che hai verso il coro e \textbf{Gigi}, la tigre di matematica. \\

\newpage

Grazie al gruppo \textit{Crenaz}.

Grazie \textbf{Spit} per l'amicizia che è nata in questi anni in mezzo a tutta la fatica. Non aver paura a tirare fuori tutto di te stesso con li amici che ti vogliono bene. Grazie \textbf{Furli} che ti prendi cura di lui.

Grazie \textbf{Ted} per l'amicizia dei primi anni ancora viva.

Grazie \textbf{Benaz} per come ti sei affezionato a me. \\


Grazie a \textbf{Dino} e alle matricole \textbf{Matte}, \textbf{Jonny}, \textbf{Diego}, \textbf{Gio Zanna}, \textbf{Mauro} e agli astronomi \textbf{Lelle}, \textbf{Cami} e \textbf{Giuli}.

Grazie \textbf{Daki} per la semplicità con cui hai iniziato da subito il lavoro dei CP nel tuo corso.

Grazie \textbf{Waka} per la serietà con cui condividi le tue fatiche. Non aver paura a buttarti.

Grazie \textbf{Ali} per la serietà e la fedeltà con cui ti sei fidata dell'amicizia con Angi. Lasciati stupire e cambiare da questa amicizia senza ideologie.

Grazie \textbf{Angi} per quanto ti sei fidata quest'anno nello studio e non. Non smettere di camminare in questa storia comunque vadano le cose e ricorda che nessuno può rispondere al tuo posto, il cammino è personale!

Grazie \textbf{Cancel} per con me ti sei aperto con me in questi anni e fidato degli amici che hai incontrato.

Grazie \textbf{Dile} per l'amicizia nata in questi ultimi anni, l'ironia e l'intelligenza. Non dimenticare che \textit{tutti i capelli del nostro capo sono contati}.

Grazie \textbf{Sabri} per l'amicizia che è nata in questi ultimi mesi. È commovente la totalità con cui ti dedichi alle cose belle che hai incontrato nella tua vita. Non smettere di spenderti in modi sempre nuovi nella compagnia che hai incontrato qui a Bologna.

\begin{quote}
    \emph{[...] Cristo, questo è il nome che indica e definisce una realtà che ho incontrato nella mia vita. Ho incontrato: ne ho sentito parlare prima da piccolo, da ragazzo, ecc. Si può diventare grandi e questa parola è risaputa, ma per tanta gente non è incontrato, non è realmente sperimentato come presente; mentre Cristo si è imbattuto nella mia vita, la mia vita si è imbattuta in Cristo proprio perché io imparassi a capire come Egli sia il punto nevralgico di tutto, di tutta la mia vita. \emph{È la vita della mia vita, Cristo.} In Lui si assomma tutto quello che io vorrei, tutto quello che io cerco, tutto quello che io sacrifico, tutto quello che in me si evolve per amore delle persone con cui mi ha messo.} \\
    Luigi Giussani, \textit{``Dare la vita per l'opera di un altro''}, 2021 BUR Rizzoli
\end{quote}

% GLI AMICI DI ADESSO E AUGURIO PER IL FUTURO


Grazie ai \textit{CP} di Scienze (e anche ai centrali):

Grazie \textbf{EleRò} per la chiarezza nel giudizio che hai sempre avuto.

Grazie \textbf{Gava} per la dedizione e cura totali che hai sempre messo nei CP e nei Social/Grafici.

Grazie \textbf{Lenny} per la semplicità e la chiarezza nel giudizio dell'attuale vita dei CP di Scienze, per l'impegno nel macello che è il Dipartimento di Matematica e per la cura nell'accogliere le matricole.

Grazie \textbf{BennyCovili} per la semplicità con cui porti tutte le tue questioni a cena con noi.

Grazie \textbf{Godz} (?), no scherzo, grazie per l'amicizia che è nata negli ultimi anni e per la serietà con cui possiamo parlare di tecnologia e scienza come di noi stessi.

Grazie \textbf{Mary Peroni} per aver detto sì a questa proposta che è un'occasione per andare più a fondo dell'amicizia che vivi a Scienze, in facoltà e fuori.

Grazie a \textbf{LetiBlack} per aver detto sì perché è uno spettacolo ultimamente vedere come ti muovi e la chiarezza nel giudizio e affezione a questa compagnia che hai. Tante cose bellissime sono ancora da scoprire. È per il centuplo quaggiù che siamo insieme! \\


Grazie alla \textit{Diaconia} di Scienze. Per me è sempre un'occasione preziosa per non essere solo nel giudicare quello che vivo e viviamo assieme e non muovermi da solo.

Grazie \textbf{Iddu} perché così almeno tutti abbiamo qualcuno da insultare. A parte gli scherzi, grazie per la serietà con cui hai sempre guardato alla Comunità di Scienze come alla tua vita. Grazie per aver condiviso sempre quello che vivi con chi ti sta intorno e per farmi notare sempre (anche in modi discutibili) quando e come sbaglio sia per come mi muovo sia per il mio carattere.

Grazie \textbf{Giaz} per la cura e disponibilità che hai avuto verso agraria e la nostra compagnia in questi anni.

Grazie \textbf{Betta} per la tua reale santità in terra! E per tutto l'aiuto di questi anni.

Grazie \textbf{Monto} (perché a parte il tuo carattere milanese sei un cucciolone) per la serietà dei nostri dialoghi e la semplicità con cui ti prendi cura della comunità di Scienze. 

Grazie \textbf{AnnaGo} per aver accettato la responsabilità della comunità, per i nostri dialoghi, pochi e brevi, ma intensi e per la serietà con cui guardi alle tue questioni. \\


Grazie \textbf{Perry} perché sei un fuoco acceso (a volte troppo) e per la gratitudine che hai verso gli amici che hai incontrato a Fisica.

Grazie \textbf{Gabri} (o dovrei dire Carlo Gabrielli) per la grande amicizia che è nata e che non credevo possibile (per motivi anagrafici). Grazie per la serietà con guardi tutto, gli altri e la tua vita, e per l'impegno col Movimento.

Grazie \textbf{Giro} per la serietà e totalità con cui vivi la tua vita. Per me sei un punto di riferimento chiaro che desidero seguire nei prossimi mesi.

Grazie \textbf{Cerne} perché nonostante anni di scontro e fatica, quando non ti nascondi dietro rabbia, odio e violenza, hai un cuore enorme e una cura e attenzione verso il prossimo che ognuno desidererebbe da un amico.

Grazie \textbf{Antuan} per la nostra complicità sulla tecnologia e nella vita. Perché uno così simile a me che ha già fatto la mia stessa strada è un grandissimo aiuto nello studio come nella vita. Aiutiamoci a giudicare insieme quello che viviamo.

Grazie \textbf{EleMariotti} per l'affetto e la compagnia che mi hai fatto in questi mesi e settimane. Continuiamo a camminare insieme. Non aver paura a dare tutto per quello che hai incontrato e stai rincontrando.

Grazie \textbf{EleCalz} per la serietà e l'estrema onestà con cui hai sempre guardato tutto e tutti e perché non ti tiri indietro dal farmi notare le cose che non vanno.

Grazie \textbf{Sbrembo} per la serietà con cui ti sei sempre guardato e hai condiviso quello che hai vissuto in questi anni. Non aver paura a dare tutto per l'amicizia che hai incontrato.

Grazie \textbf{Sofy} per l'aiuto nello studio di questi anni e per la compagnia umana e nei giudizi. Ripartiamo dall'origine della nostra amicizia che è Qualcosa più grande di noi.

Grazie \textbf{Jack Dipa} per la serietà con cui mi hai guardato fin dal primo istante. Ci sono voluti anni per fare i conti col tuo carattere estroverso, esplosivo e diretto, ma alla fine ha avuto la meglio la serietà e intelligenza dei tuoi giudizi.

Grazie \textbf{\TeX} per la compagnia e l'aiuto nel giudizio in questi anni. La tua chiarezza è sempre stato motivo di grande stima per me. (Grande \textbf{Ida}!) Sei come un padre per me.

Grazie \textbf{Santo}. Non bastano le parole per esprimere la stima e la gratitudine nei tuoi confronti per l'aiuto e la compagnia di tutti questi anni. Dal primo giorno ti sei preso cura di me come di un figlio, sempre indicandomi la strada secondo te giusta quando te lo chiedevo, ma sempre lasciandomi libero di sbagliare. Grazie per aver condiviso con me il cammino delle elezioni. Mi spiace che ci siamo allontanati il mio secondo anno, ma è stata una gioia ripartire dall'origine del nostro rapporto. Grazie per l'aiuto per questa tesi che senza di te probabilmente non ci sarebbe stata.

\begin{verse}
    \textit{È lunga questa notte l'avventura \\
    e l'autostrada non finisce mai, \\
    penso a tutte le cose che ho avuto, \\
    penso a tutte le cose che mi dai.}
    
    \textit{La nebbia adesso non mi fa paura \\
    e immagino i bambini addormentati, \\
    anche stanotte torno, stai sicura, \\
    il giorno ci ritroverà abbracciati.}
    
    \textit{Penso a tutti gli amici che ho incontrato, \\
    a quelli che non ho saputo amare, \\
    a tutte le canzoni che ho cantato \\
    e a te che non ti stanchi di aspettare.}
    
    \textit{È bella la fatica del lavoro, \\
    la contentezza non finisce mai, \\
    penso a tutte le cose che mi hai dato, \\
    penso a tutte le cose che mi dai.}
    
    \textit{I miei passi diventano pensieri \\
    e i pensieri diventano Qualcuno, \\
    diventano Te, Padre grande e buono, \\
    che per amore hai cominciato il gioco.}
    
    \textit{Non lasciare che un giorno me ne vada, \\
    dammi sempre la forza di lottare, \\
    è ancora molto lunga questa strada \\
    e ho ancora tanta voglia di cantare:}
    
    \textit{lalalalalalalalalala \\
    lalalalalalalalalala \\
    è ancora molto lunga questa strada \\
    e ho ancora tanta voglia di cantare.}
    
    \textit{``Canzone per te''}, Claudio Chieffo, novembre 1985
\end{verse}

Con l'augurio di continuare a camminare tutti insieme verso un Destino buono. \\
Grazie di cuore a tutti!