%-----------------------------CONCLUSIONI
Nel presente lavoro di tesi è stato implementato un programma di Machine Learning (ML) in grado di addestrare reti neurali convoluzionali (CNN) da utilizzare per analisi nel campo della fisica delle alte energie (High Energy Physics, HEP). Per l’implementazione software sono state utilizzate le librerie open source TensorFlow e l’API Keras. Il programma, scritto con il linguaggio di programmazione Python, comprende diversi strumenti per l’analisi preliminare delle variabili di ingresso, l’addestramento delle reti neurali e per il calcolo delle metriche per la valutazione della qualità del training. 

Una volta completato, il programma è stato testato con dati forniti dall’esperimento ALICE riguardanti il barione $\Lambda_{c}^{+}$ ed in particolare il suo decadimento adronico $\Lambda_{c}^{+} \to p K^{0}_{S}$. Il campione di segnale è stato prodotto da simulazioni di collisioni $pp$ ad una energia del centro di massa di $\sqrt{s} = $ \qty{13}{\tera \eV} prodotte con PYTHIA8 e GEANT3, mentre per il campione di fondo si sono utilizzati dati sperimentali raccolti dall’esperimento ALICE utilizzando candidate con una massa invariante ricostruita non compatibile con la massa di un barione $\Lambda_{c}^{+}$. L’analisi si è concentrata nell’intervallo di impulso trasverso $1~<~p_{T}~<~2$~\unit{\giga \eV \per \clight}.

I risultati ottenuti con questo test dimostrano la capacità dell’algoritmo implementato di addestrare in maniera adeguata il modello di ML. Per finalizzare la scrittura del framework, il prossimo step sarà l’implementazione dell’algoritmo che si occupa di applicare il modello addestrato ai dati reali. Tale framework fornirà un tool potente, flessibile e personalizzabile, da utilizzare non solo per la ricostruzione del barione charmato $\Lambda_{c}^{+}$ ma in generale per ogni tipo di analisi nel campo della fisica delle alte energie.

\begin{comment}
Nel presente lavoro di tesi il barione charmato $\Lambda^{+}_{c}$ è stato ricostruito attraverso il suo decadimento $\Lambda_{c}^{+} \to p K^{0}_{S}$ utilizzando i dati raccolti dall’esperimento ALICE a LHC in collisioni $pp$ ad una energia del centro di massa di $\sqrt{s} =$ \qty{13}{\tera \eV}. L’analisi si è concentrata nell’intervallo di impulso trasverso $1 < pT < 2$ \unit{\giga \eV \per \clight} in quanto a basso $p_{T}$ i vari modelli teorici mostrano una certa differenza nelle loro previsioni ed è quindi possibile, attraverso l’analisi dei dati sperimentali, discriminare l’attendibilità di tali previsioni così come fornire correzioni ai vari modelli.

La ricostruzione ha fatto utilizzo di tecniche di Machine Learning per mezzo dell'allenamento di una Rete Neurale.

Il codice prodotto ha permesso di creare un framework di analisi dati con le libreire open source di Keras e TensorFlow implementando una Neural Network potenzialmente applicabile a \textit{qualsiasi} analisi, di High Energy Physics e non, purché binaria, ovvero in una situazione simile a quella di segnale (1) e fondo (0).

Sono state inserite diverse metriche al fine di valutare la bontà del training e i risultati confermano che il programma è correttamente funzionante.

Adesso che la rete è stata allentata sui dati simulati che fornivano segnale e fondo separatamente, è possibile metterla al lavoro per separare il segnale dal fondo sui dati reali acquisiti dall'esperimento ALICE. Questi dati sono una combinazione di segnale e fondo difficilmente affrontabile non disponendo di tecniche computazionali e di Machine Learning, Pertanto è indispensabile l’utilizzo di software in grado di rigettare una parte considerevole del fondo combinatoriale mantenendo allo stesso tempo una buona efficienza di selezione.
\end{comment}