%-------------------------------------FONT
\usepackage[T1]{fontenc}
\usepackage[utf8]{inputenc}
\usepackage{newlfont} % per frontespizio
%\usepackage{mathpazo} % font ArsClassica

%-------------------------------------MATEMATICA
\usepackage{amsmath}
\usepackage{amsfonts}
\usepackage{amssymb}
\usepackage[output-decimal-marker={.}]{siunitx}
% siunitx: numeri con unità di misura
% output-decimal-marker={,}: le convenzioni tipografiche italiane prevedono la virgola e non il punto

%-------------------------------------LINGUE
\usepackage[english,
            italian]{babel}
\usepackage[autostyle,
            italian=guillemets]{csquotes} % per virgolette corrette
% autostyle adatta lo stile delle citazioni alla lingua corrente del documento
% italian=guillemets racchiude automaticamente tra virgolette caporali i campi che prevedono le virgolette


%-------------------------------------IMMAGINI
\usepackage{graphicx}
% serve per includere immagini e grafici
\graphicspath{{res/fig/}}
% importa la cartella res/fig/ come cartella da cui caricare le immagini

\usepackage[inkscapelatex=false]{svg}
% permette di caricare immagini svg

% \usepackage{flafter}
% impedisce alle figure di apparire prima della loro definizione nel testo

%\usepackage{float}
% permette di forzare il posizionamento dell’oggetto nel punto in cui è situato con l’opzione H

%-------------------------------------BORDI
\usepackage{geometry} % permette la modifica della gabbia del documento
\geometry{
  a4paper, % formato di pagina
  heightrounded, % modifica di poco le dimensioni della gabbia per contenere un numero intero di righe
  hmargin=2.6cm, % dimensioni margini destro-sinistro
  %vmargin=2.5cm % dimensioni margini superiore-inferiore
}

%\usepackage{layaureo}
% Pacchetto layaureo per la formattazione italiana

%-------------------------------------RIFERIMENTI
\usepackage{hyperref}

%ANTO
%\usepackage[colorlinks=true,
 %   linkcolor=black,
  %  urlcolor=teal,
   % citecolor=black]{hyperref}

%\hypersetup{
 %   colorlinks=true,
  %  linkcolor=black,
   % filecolor=magenta,      
    %urlcolor=cyan,
    %pdftitle={Overleaf Example},
    %pdfpagemode=FullScreen,
%}

%-------------------------------------BIBLIOGRAFIA
\usepackage[backend=biber,
            bibstyle=numeric,
            citestyle=numeric,
            backref,
            hyperref]{biblatex}
% Biber come motore bibliografico, è nuovo e da preferire a BibTex
% backref: indica accanto a ciascun riferimento le pagine del documento in cui è citato
\addbibresource{biblio.bib}

%-------------------------------------ALTRI
%\usepackage{setspace} % serve a fornire comandi di interlinea standard
%\onehalfspacing{} % imposta interlinea a 1,5 ed equivale a \linespread{1,5}

\usepackage{xcolor} % colori frontespizio in bozza

%\usepackage[a-1b]{pdfx} % conformità pdf generato

%\usepackage{lineno}
%\linenumbers

\usepackage{comment}

\usepackage{fancyhdr}
% per intestazioni e più di pagina (fancy header)

\newcommand{\fncyfront}{%
    \fancyhead[RO]{{\footnotesize\rightmark}}
    \fancyfoot[RO]{\thepage}
    \fancyhead[LE]{\footnotesize{\leftmark}}
    \fancyfoot[LE]{\thepage}
    \fancyhead[RE,LO]{}
    \fancyfoot[C]{}
    \renewcommand{\headrulewidth}{0.3pt}}
\newcommand{\fncymain}{%
    \fancyhead[RO]{{\footnotesize\rightmark}}
    \fancyfoot[RO]{\thepage}
    \fancyhead[LE]{{\footnotesize\leftmark}}
    \fancyfoot[LE]{\thepage}
    \fancyfoot[C]{}
    \renewcommand{\headrulewidth}{0.3pt}}

\newenvironment{abstract}
% definizione ambiente abstract ispirato ad article non presente in book
  {%\cleardoublepage% togliere primo % se non si vuole sommario come un chapter
    \thispagestyle{empty}%
    \null \vfill\begin{center}%
      \chapter*{\abstractname} \end{center}}%
  {\vfill\null}

%-------------------------------------Definizioni di comandi e ambienti